% Resume Template 
% Created by Jean Sung
% Summer 2014
% Last updated : October 2014
% find updates / submit feedback
% @ https://github.com/jeansung/ResumeTemplate


% % Instructions & how to use
% 1) fill in resume header details 
% 2) create resume by using different section styles
% 3) copy and paste section as needed and fill in
%    relevant information
% Notes:
% -default font is 12pt, can be changed at top of LaTex
%	document, but remember that a different font means
%   different places for newlines if you don't want what's
%	flush with the left to interfere with date column
% -default setting for date column indentation explained 
%	below, can be changed

% % Start of Document % %
\documentclass[12pt]{article}
\usepackage[margin= 0.5 in]{geometry}
\usepackage{hyperref}
\pagenumbering{gobble}
\normalsize

% % resume header details % %
\newcommand{\name}{\large\textbf{Gourav Khadge}} % Lovely sample name 
\newcommand{\addr}{\small 1 Rocket Road}
\newcommand{\email}{\small \url{engineer@techschool.edu}}
\newcommand{\phone}{\small phone number }

% % Information about Indent for Date Colum % %
% 8.5 inch wide paper (assuming standard US paper size)
% margin = 0.5 inch (can be changed at the top)
% date = 1 inch 
% 8.5 - margin - date Indent = indent for dates [6.5]
\newcommand{\aligndates}{\hspace*{6.5in}}

% % shortcuts & their descriptions % %
% header dot 
% used to seperate address, email and phone on
% the header of the document
\newcommand{\headerdot}{  $\bullet$  }

% vb
% used to seperate items in Section Style Gamma
\newcommand{\vb}{ $\mid$ }

% sectionNL the newline after a section title
\newcommand{\sectionNL}{\\[2pt]}

% rightalign right aligns the dates of events for better
% aesthetics
\newcommand{\rightalign}{\hfill}

% custom tab & custom tabinline
% for use with section style beta
% the first line of a detail in beta is prefixed with \customtab
% any subsequent lines of a detail start with a \customtabinline
% after the newline command 
\newcommand{\customtab}{$\hspace{10pt}\bullet\hspace{2pt}$}
\newcommand{\customtabinline}{$\hspace{17pt}$}
\begin{document}

% Contact Information 
\begin{center}
\name \\
\addr \headerdot \email \headerdot \phone
\end{center}

% Section Style Alpha
% Suggested uses: Education, Awards, Honors, Activities
% Template:
\begin{flushleft}
{\textbf{Section Alpha Title}}  \sectionNL
Subsection category 1 \rightalign Month Year \\
Second line   \\~\\
Subsection category 2 \rightalign season \\
Second Line
\end{flushleft}


% Section Style Beta
% Suggested uses: research / work/ volunteer experience
% Template: 
\begin{flushleft}
{\textbf{Section Beta Title}}  \sectionNL
Activity Title 1, Location \rightalign Dates \\
\customtab Detail 1   \\~\\

Activity Title 2 \rightalign Dates \\
\customtab Detail 2 example is really quite long because there are a lot of things I want to say \\ \customtabinline about this activity spills to the new line by using the customtablinline command
\end{flushleft}


% Section Style Gamma
% Suggested uses: relevant coursework 
% Template: 
\begin{flushleft}
{\textbf{Section Gamma Title }} \sectionNL
Item 1 \vb Item 2 \vb Item 3, etc 
\end{flushleft}

% Section Style Delta
% Suggested uses: Skills list 
% Template: 
\begin{flushleft}
{\textbf{Section Delta Title}} \sectionNL
\textit{Subsection 1:} item 1, item 2, item 3, item 4, etc
\end{flushleft}

\end{document}
